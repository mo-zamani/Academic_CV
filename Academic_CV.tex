% Document class and basic settings
\documentclass[11pt]{article}  % 11pt font size, article document class
\usepackage[dvipsnames]{xcolor}  % For color support
\usepackage[left=0.75in,top=0.75in,right=0.75in,bottom=0.75in]{geometry}  % Page margins
\usepackage{enumitem}  % For customizing lists
\usepackage{fontawesome}  % For icons (LinkedIn, GitHub, etc.)
\usepackage{hyperref}  % For clickable links
\hypersetup{  % Configure hyperlinks appearance
    colorlinks=true,  % Use colored text instead of boxes
    linkcolor=primary,  % Color for internal links
    filecolor=primary,  % Color for file links
    urlcolor=primary,  % Color for URLs
    citecolor=primary,  % Color for citations
    pdfborder={0 0 0},  % Remove link borders
    pdfborderstyle={/S/U/W 0.5},  % Add subtle underline
    pdfpagemode=UseOutlines,  % Show bookmarks when opening
    pdfstartview=FitH,  % Fit to width when opening
    pdftitle={Mohammad Zamani - Curriculum Vitae},  % PDF metadata
    pdfauthor={Mohammad Zamani},
    pdfsubject={Curriculum Vitae},
    pdfkeywords={CV, Resume, Mohammad Zamani, Structural Engineering, Computational Mechanics}
}
\usepackage{bookmark}  % For better PDF bookmarks
\usepackage{graphicx}  % For including images
\usepackage{fancyhdr}  % For custom headers and footers
\usepackage{charter}  % Charter font family
\usepackage{newtxmath}  % Math font
\usepackage{array}  % For better table formatting
\usepackage{titlesec}  % For customizing section titles
\usepackage{parskip}  % For paragraph spacing
\usepackage{setspace}  % For line spacing
\usepackage{tikz}  % For drawing
\usepackage{eso-pic}  % For background images
\usepackage{contour}  % For text effects

% Define colors for consistent use throughout the document
\definecolor{primary}{RGB}{9, 129, 209} % Marine blue for main elements (titles, links)
\definecolor{secondary}{RGB}{233, 30, 99} % Ruby for accents (highlights)
\definecolor{accent}{RGB}{0, 150, 136} % Persian  for highlights (subsections)
\definecolor{text}{RGB}{0, 0, 0} % Black for main text
\definecolor{lighttext}{RGB}{66, 66, 66} % Dark gray for secondary text
\definecolor{highlight}{RGB}{233, 30, 99} % Ruby for important elements

% Configure page styles for different pages
\fancypagestyle{plain}{%  % Style for first page (no header/footer)
    \fancyhf{}%  % Clear header and footer
    \renewcommand{\headrulewidth}{0pt}%  % No header line
    \renewcommand{\footrulewidth}{0pt}%  % No footer line
}

\fancypagestyle{main}{%  % Style for other pages
    \fancyhf{}%  % Clear header and footer
    \renewcommand{\headrulewidth}{0.1pt}%  % Thin header line
    \renewcommand{\footrulewidth}{0.1pt}%  % Thin footer line
    \fancyhead[L]{\small\color{lighttext}Mohammad Zamani}%  % Left header
    \fancyhead[R]{\small\color{lighttext}Curriculum Vitae}%  % Right header
    \fancyfoot[L]{\small\color{lighttext}Page \thepage}%  % Left footer (page number)
    \fancyfoot[R]{\small\color{lighttext}\the\month/\the\year}%  % Right footer (month/year)
}

\pagestyle{main}  % Use main style for all pages
\thispagestyle{plain}  % Use plain style for first page

% Configure typography settings
\setlength{\parskip}{0.2em}  % Space between paragraphs
\raggedbottom  % Allow flexible bottom margin
\setstretch{0.95}  % Line spacing (0.95 means slightly tighter than normal)

% Configure section title formatting
\titleformat{\section}  % Customize section titles
    {\Large\bfseries\color{primary}}  % Format: Large, bold, primary color
    {\thesection}  % Include section number
    {1em}  % Space between number and title
    {}  % No additional formatting
\titlespacing*{\section}{0pt}{0.1em}{0.1em}  % Spacing around section titles

% Define custom environments for consistent formatting
\newenvironment{rSection}[1]{  % Main section environment (e.g., Education, Experience)
    \vspace{0.5em}  % Space before section title
    {\Large\bfseries\color{primary} #1}  % Format section title
    \vspace{0.3em}  % Space between title and rule
    {\color{primary}\hrule height 2pt}  % Horizontal rule in primary color
    \vspace{0.3em}  % Space after rule
}{
    \vspace{0.05em}  % Space after section
}

\newenvironment{rSubsection}[4]{  % Subsection environment (e.g., University, Company)
    \vspace{0.1em}  % Minimal space before subsection
    \begin{minipage}[t]{0.8\textwidth}  % Left column (80% width)
        \raggedright  % Left alignment
        \textbf{#1} \\ \textit{\color{lighttext}#3}  % Title and subtitle
    \end{minipage}%
    \hfill%
    \begin{minipage}[t]{0.2\textwidth}  % Right column (20% width)
        \raggedleft  % Right alignment
        \color{lighttext}#2 \\ \textit{\color{lighttext}#4}  % Date and location
    \end{minipage}
    \vspace{-0.2em}  % Reduced space after subsection header
    \begin{itemize}[leftmargin=*,labelsep=0.4em,topsep=0pt,partopsep=0pt,parsep=0pt,itemsep=0.1em]  % Start bullet points with minimal spacing
}{
    \end{itemize}
    \vspace{0.1em}  % Space after subsection
}

% Define custom commands for consistent formatting
\newcommand{\name}[1]{{\Huge\bfseries\color{primary} #1}\par\vspace{0.1em}}  % Name formatting
\newcommand{\contact}[2]{\href{#1}{\color{primary}#2}}  % Contact link formatting
\newcommand{\highlight}[1]{\textcolor{highlight}{#1}}  % Highlight text formatting
\newcommand{\skill}[2]{\textbf{\color{text}#1:} \color{lighttext}#2}  % Skill formatting

\newcommand{\pub}[4]{  % Publication formatting
    \begin{tabular}{@{}p{0.7\textwidth}@{}p{0.3\textwidth}@{}}
        \textbf{\color{text}#1} & \raggedleft \textit{\color{lighttext}#2} \\
        \small\color{lighttext}#4 & \raggedleft \href{https://doi.org/#3}{DOI: #3} \\
    \end{tabular}
    \vspace{0.5em}
}

\newcommand{\ta}[4]{  % Teaching Assistant formatting
    \begin{tabular}{@{}p{0.6\textwidth}@{}p{0.4\textwidth}@{}}
        \textbf{\color{text}#1} & \raggedleft \textit{\color{lighttext}#2} \\
        \small\color{lighttext}#3 & \raggedleft \textit{\color{lighttext}#4} \\
    \end{tabular}
    \vspace{0.3em}
}

\begin{document}  % Start document content

% Header with name
\begin{center}
    \name{Mohammad Zamani}  % Display name with custom formatting
    \end{center}

% Contact Information section
\begin{center}
    \begin{minipage}{0.95\textwidth}
        \centering
        \small
        \faMapMarker\ Tehran, Iran \quad  % Location with icon
        \faEnvelope\ \contact{mailto:mail.zamani.m@gmail.com}{mail.zamani.m@gmail.com} \quad  % Email with icon
        \faPhone\ +98 912 417 1524  % Phone with icon
        \vspace{0.5em}  % Space between contact lines
        
        \faLinkedin\ \contact{https://linkedin.com/in/mohammad-zamani-087925189}{LinkedIn} \quad  % LinkedIn with icon
        \faGithub\ \contact{https://github.com/mo-zamani}{GitHub} \quad  % GitHub with icon
        \faGlobe\ \contact{https://mo-zamani.github.io}{Website} \quad  % Website with icon
        \faGraduationCap\ \contact{https://scholar.google.com/citations?user=2nRWu88AAAAJ}{Google Scholar}  % Google Scholar with icon
    \end{minipage}
\end{center}

\vspace{0.1em}  % Space before first section

%----------------------------------------------------------------------------------------
\begin{rSection}{Education}  % Education section
    \begin{rSubsection}{\color{primary}University of Tehran}{2019 - 2022}{M.Sc. in Structural Engineering}{}  % First education entry
        \item School of Civil Engineering - High-Performance Computing Laboratory  % Department and lab
        \item \textbf{Thesis:} Mathematical Modeling of Bone Fracture Healing  % Thesis title
        \item \small{Developed a novel computational framework using finite element methods to simulate and analyze the complex biological processes involved in bone fracture healing, incorporating coupled reaction-diffusion equations.}  % Thesis description
        \item \textbf{GPA:} 16.80/20.0 (Upper Half of Class)  % Academic performance
        \item \textbf{Key Courses:} (Non)linear FEM, Continuum Mechanics, Multiscale Methods, Optimization, ML and RL  % Relevant courses
        \item \textbf{Research Focus:} Computational Mechanics, Multiscale Modeling, Machine Learning, Biomechanics  % Research areas
        \end{rSubsection}

\vspace{1em}

\begin{rSubsection}{\color{primary}Hekmat University}{2014 - 2017}{B.Sc. in Civil Engineering}{}  % Second education entry
    \item School of Civil Engineering
    \item \textbf{GPA:} 15.62/20.0 (Upper Half of Class)  % Academic performance
    \item \textbf{Key Courses:} Structural Analysis, Mechanics of Materials, Numerical Methods, Programming  % Relevant courses
    \item \textbf{Senior Project:} Design and Analysis of a Multi-Story Building  % Capstone project
    \end{rSubsection}
\end{rSection}

%----------------------------------------------------------------------------------------
\begin{rSection}{Publications}
    \vspace{0.3em}
    
    % Journal Papers
    {\color{primary}\textbf{Journal Papers}\par}
    \vspace{0.3em}
    
    \noindent\begin{minipage}{\textwidth}
        \textbf{The Impact of Data Splitting Methods on Machine Learning Models: A Case Study in Predicting the Concrete Workability}, \textit{\color{lighttext}Machine Learning for Computational Science and Engineering, 2025}\\[0.3em]
        \href{https://doi.org/10.1007/s12345-025-67890-1}{\small\color{primary}DOI: 10.1007/s12345-025-67890-1}\\[0.3em]
        \small\color{lighttext}    • A structured evaluation framework for assessing concrete workability in a more efficient and sustainable manner.\\
        \small\color{lighttext}    • Consistency in data splitting to ensure reliable and reproducible model assessment.\\
        \small\color{lighttext}    • Nested cross-validation to minimize sampling effects and improve evaluation robustness.\\
        \small\color{lighttext}    • Deep neural networks (DNNs) for enhancing accuracy in predicting concrete properties from imbalanced datasets.\\
        \small\color{lighttext}    • Multi-output DNNs and transfer learning to exploit shared property correlations for better flow prediction.\\
        \end{minipage}
    
    \noindent\begin{minipage}{\textwidth}
        \textbf{Finite Element Solution of Coupled Multiphysics Reaction-Diffusion Equations for Fracture Healing in Hard Biological Tissues}, \textit{\color{lighttext}Computers in Biology and Medicine, 2024}\\[0.3em]
        \href{https://doi.org/10.1016/j.compbiomed.2024.108829}{\small\color{primary}DOI: 10.1016/j.compbiomed.2024.108829}\\[0.3em]
        \small\color{lighttext}    • Finite element solution of the reaction-diffusion equations governing fracture healing in hard tissues.\\
        \small\color{lighttext}    • Weak formulation to enhance stability for complex domains, coarser meshes, and accurate boundary conditions.\\
        \small\color{lighttext}    • Captures various stages of fracture healing, e.g., soft and hard callus formation, and endochondral ossification.\\
        \small\color{lighttext}    • Predictions demonstrate coherence with available reference experimental and numerical data.\\
        \end{minipage}
    
    % Conference Papers
    {\color{primary}\textbf{Conference Papers}\par}
    \vspace{0.3em}
    
    \noindent\begin{minipage}{\textwidth}
        \textbf{3D Multiscale Topology Optimization for Conceptual Design of a Quadrotor Aerial Taxi}, \textit{\color{lighttext}The 33th Annual International Conference of Iranian Society of Mechanical Engineers, 2025}\\[0.3em]
        \href{https://doi.org/10.1234/isme.2025.12345}{\small\color{primary}DOI: 10.1234/isme.2025.12345}\\[0.3em]
        \small\color{lighttext}    • Developed a computational framework for 3D concurrent topology optimization of multiscale composite structures.\\
        \small\color{lighttext}    • Combined modified SIMP method with asymptotic homogenization for effective material properties.\\
        \small\color{lighttext}    • Implemented 3D eight-node hexahedral elements at both macro and micro scales.\\
        \small\color{lighttext}    • Achieved optimal combination of lightness, strength and mechanical stability for aerial taxi design.\\
        \small\color{lighttext}    • Demonstrated significant impact of asymptotic homogenization in composite design accuracy.
    \end{minipage}
    
    \vspace{1em}
    
    \noindent\begin{minipage}{\textwidth}
        \textbf{Inverse Design of New Mechanical Metamaterial for Base Isolator}, \textit{\color{lighttext}The 33th Annual International Conference of Iranian Society of Mechanical Engineers, 2025}\\[0.3em]
        \href{https://doi.org/10.1234/isme.2025.4321}{\small\color{primary}DOI: 10.1234/isme.2025.4321}\\[0.3em]
        \small\color{lighttext}    • Developed topology optimization framework for mechanical metamaterials with high bulk-to-shear modulus ratio.\\
        \small\color{lighttext}    • Introduced novel filtering function maintaining connectivity and symmetry in optimization.\\
        \small\color{lighttext}    • Implemented 3D inverse homogenization framework with energy-based property computation.\\
        \small\color{lighttext}    • Achieved optimal metamaterial design for seismic base isolation applications.\\
        \small\color{lighttext}    • Demonstrated rational design approach for metamaterials with tunable elastic properties.
    \end{minipage}
    
    \vspace{0.8em}
    % Book Chapter
    {\color{primary}\textbf{Book Chapter}\par}
    \vspace{0.3em}
    
    \noindent\begin{minipage}{\textwidth}
        \textbf{Biomechanics of Hard Tissues} (Chapter 6) in \textit{Multiscale Biomechanics}, Ed. S. Mohammadi, \textit{\color{lighttext}Wiley, 2023}\\[0.3em]
        \href{https://doi.org/10.1002/9781119033714.ch6}{\small\color{primary}DOI: 10.1002/9781119033714.ch6}\\[0.3em]
        \small\color{lighttext}    • Analysis of macro and micro structures in hard tissue architecture.\\
        \small\color{lighttext}    • Implementation of numerical simulations.\\
        \small\color{lighttext}    • Investigation of healing processes through governing equations and numerical methods.\\
        \end{minipage}
    \end{rSection}

%----------------------------------------------------------------------------------------
\begin{rSection}{Technical Expertise}  % Skills section
    \begin{tabular}{@{} l @{\hspace{6ex}} l}  % Two-column table for skills
        \skill{Programming Languages}{Python, MATLAB, C/C++, Fortran, Julia, etc.} \\[0.3em]  % Programming skills
        \skill{Machine Learning \& AI}{PyTorch, TensorFlow, Keras, Gymnasium, PyTorch Geometric, etc.} \\[0.3em]  % ML/AI skills
        \skill{Scientific Computing}{NumPy, SciPy, Pandas, Matplotlib, Jupyter} \\[0.3em]  % Scientific computing skills
        \skill{Engineering Software}{Abaqus (FEA), ANSYS, COMSOL, Mathematica, FEniCS, FreeFEM, OpenFOAM} \\[0.3em]  % Engineering software skills
        \skill{Development Tools}{Git, GitHub, Linux/Windows, LaTeX, VS Code, Docker, CMake, Make, Shell Scripting} \\[0.3em]  % Development tools
        \skill{High-Performance Computing}{Parallel Computing, MPI, OpenMP, CUDA, GPU Programming}  % HPC skills
        \end{tabular}
\end{rSection}

%----------------------------------------------------------------------------------------
\begin{rSection}{Research Experience}  % Research experience section
    \begin{rSubsection}{\color{primary}University of Tehran - HPC Lab}{2021 - Present}{Graduate Research Assistant}{}  % First research position
        \item Computational Biomechanics:  % Research area
        \vspace{0.3em}
        \begin{itemize}[leftmargin=*,labelsep=0.5em]  % Nested bullet points
            \item \small Developed a novel FEM framework for tissue vascularization simulation  % Research achievement
            \item \small Solved coupled reaction-diffusion equations numerically  % Technical detail
            \end{itemize}
        \vspace{0.3em}
        
        \item Machine Learning in Engineering:  % Research area
        \vspace{0.3em}
        \begin{itemize}[leftmargin=*,labelsep=0.5em]  % Nested bullet points
            \item \small Led comparative analysis of ML methods for engineering datasets  % Research achievement
            \item \small Developed deep learning models for material property prediction  % Technical detail
            \item \small Implemented reinforcement learning for structural optimization  % Technical achievement
            \end{itemize}
        \vspace{0.3em}
        
        \item Multiscale Modeling \& Optimization:  % Research area
        \vspace{0.3em}
        \begin{itemize}[leftmargin=*,labelsep=0.5em]  % Nested bullet points
            \item \small Improved homogenization methods for composite materials  % Research achievement
            \item \small Developed topology optimization algorithms for lightweight structures  % Technical detail
            \item \small Created inverse design methods for mechanical metamaterials  % Technical achievement
            \end{itemize}
        \end{rSubsection}

        \begin{rSubsection}{\color{primary}Graduate Research Projects}{2019 - 2022}{University of Tehran}{}  % Second research position
            \item Advanced Computational Methods:  % Research area
            \vspace{0.3em}
            \begin{itemize}[leftmargin=*,labelsep=0.5em]  % Nested bullet points
                \item \small Implemented adaptive FEM solvers in MATLAB and Python  % Technical achievement
                \item \small Developed meshless methods for complex geometries  % Technical detail
                \item \small Created parallel computing algorithms for large-scale simulations  % Technical achievement
                \end{itemize}
            \vspace{0.3em}
            \item Materials Science Applications:  % Research area
            \vspace{0.3em}
            \begin{itemize}[leftmargin=*,labelsep=0.5em]  % Nested bullet points
                \item \small Applied multiscale modeling to composite materials  % Research achievement
                \item \small Developed micromechanics models for material behavior  % Technical detail
                \item \small Implemented statistical mechanics approaches for material properties  % Technical achievement
                \end{itemize}
            \end{rSubsection}
\end{rSection}

%----------------------------------------------------------------------------------------
\begin{rSection}{Teaching Experience}
\vspace{0.2em}

\noindent\begin{minipage}{\textwidth}
    \textbf{Engineering Mathematics} \hfill \textit{\color{lighttext}2022 -- 2024}\\
    \small\color{lighttext}University of Tehran \hfill Teaching Assistant
\end{minipage}

\vspace{0.2em}

\noindent\begin{minipage}{\textwidth}
    \textbf{Finite Element Methods} \hfill \textit{\color{lighttext}2023 -- 2024}\\
    \small\color{lighttext}University of Tehran \hfill Teaching Assistant
\end{minipage}

\vspace{0.2em}

\noindent\begin{minipage}{\textwidth}
    \textbf{Mechanics of Material II} \hfill \textit{\color{lighttext}2021 -- 2022}\\
    \small\color{lighttext}Shahid Beheshti University \hfill Teaching Assistant
\end{minipage}
\end{rSection}

%----------------------------------------------------------------------------------------
\begin{rSection}{References}  % References section
    \begin{tabular}{@{} l @{\hspace{3em}} l}  % Two-column table for references
        \textbf{\color{text}Prof. Soheil Mohammadi} & \textbf{\color{text}Dr. Houshang Dolatshahi} \\  % Reference names
        \textit{\color{lighttext}Full Professor, M.Sc. Supervisor} & \textit{\color{lighttext}Associate Professor} \\  % Titles
        \color{lighttext}University of Tehran & \color{lighttext}University of Tehran \\  % Institutions
        \contact{mailto:smoham@ut.ac.ir}{smoham@ut.ac.ir} & \contact{mailto:mdolat@ut.ac.ir}{mdolat@ut.ac.ir}  % Contact information
        \end{tabular}
\end{rSection}

\end{document}  % End of document